\documentclass[12pt, titlepage]{article}

\usepackage{booktabs}
\usepackage{tabularx}
\usepackage{hyperref}
\usepackage{enumitem}
\usepackage{titlesec}
\hypersetup{
    colorlinks,
    citecolor=black,
    filecolor=black,
    linkcolor=red,
    urlcolor=blue
}
\usepackage[round]{natbib}

\setcounter{secnumdepth}{4}
\setlist[description]{leftmargin=\parindent,labelindent=\parindent}

\titleformat{\paragraph}
{\normalfont\normalsize\bfseries}{\theparagraph}{1em}{}
\titlespacing*{\paragraph}
{0pt}{3.25ex plus 1ex minus .2ex}{1.5ex plus .2ex}

\title{SE 3XA3: Test Plan\\PasswordProtectionProgram}

\author{Team 28, Tuples1
		\\ Shabana Dhayananth, dhayanas
		\\ Suhavi Sandhu, sandhs
		\\ Joseph Lu, luy89
}

\date{\today}

\begin{document}

\maketitle

\pagenumbering{roman}
\tableofcontents
\listoftables
\listoffigures

\begin{table}[bp]
\caption{\bf Revision History}
\begin{tabularx}{\textwidth}{p{3cm}p{2cm}X}
\toprule {\bf Date} & {\bf Version} & {\bf Notes}\\
\midrule
2017-10-27 & 0.0 & Creation\\
\bottomrule
\end{tabularx}
\end{table}

\newpage

\pagenumbering{arabic}

\section{General Information}

\subsection{Purpose}
This document is a rigorous guideline that encompasses the testing, validation and verification processes that the software, PasswordProectionProgram, shall undergo.

\subsection{Scope}
This test plan will be used as a outline for unit testing the encryption of user information, the GUI and the database used to store the user data. This is to facilitate the identification of software bugs in each specific section and to allow the testing to be done by each group member. These components will also be integration tested to verify that the system as a whole functions as intended.


\subsection{Acronyms, Abbreviations, and Symbols}
Refer to Table 2 and Table 3
	
\begin{table}[hbp]
\caption{\textbf{Table of Abbreviations}} \label{Table}

    \begin{tabularx}{\textwidth}{p{3cm}X}
        \toprule
        \textbf{Abbreviation} & \textbf{Definition} \\
        \midrule
        GUI & Graphical User Interface\\\hline
        OS & Operating System\\\hline
        POC & Proof of Concept, used for naming tests for proof of concept\\
        \bottomrule
    \end{tabularx}

\end{table}

\begin{table}[!htbp]
\caption{\textbf{Table of Definitions}} \label{Table}

    \begin{tabularx}{\textwidth}{p{3cm}X}
        \toprule
        \textbf{Term} & \textbf{Definition}\\
        \midrule
        Home Screen & Screen that application first opens up to, user can enter master password here to sign in to password manager\\\hline
        Password Manager Screen & Screen in which user can input and view accounts, usernames and passwords.\\\hline
        Setting Screen & Screen in which master password can be changed\\\hline
        PROCESSING-TIME & Timeframe within which system must visibly  respond to user input, 2 seconds.\\\hline
        ERROR-TIME & Timeframe within which system must visibly respond to incomplete/incorrect user input, 2 seconds.\\
        \bottomrule
    \end{tabularx}

\end{table}	

\subsection{Overview of Document}
The functional and nonfunctional requirements for which tests are determined in this document refer to a more specific version of those requirements mentioned in the Software Requirements Specification Document (SRS-Rev0). The names for the functional and non-functional test cases are prefixed with FR for functional and NFR for non-functional. The middle abbreviation for the test cases refer to the category that they are listed under and then they are suffixed with a numerical value.

\section{Plan}

\subsection{Software Description}
The PasswordProtectionProgram is a password manager that uses a Python encryption library to keep user passwords safe and holds them in a local database. The implementation for this application is in Python.

\subsection{Test Team}
The members of the team, Suhavi Sandhu, Shabana Dhayananth and Joseph Lu,  are responsible for all the procedures of the validation process including writing and executing tests.

\subsection{Testing Tools}
The only tool required for testing is PyUnit, which will be used for unit testing and integration testing.

\subsection{Testing Schedule}
Refer to Table 4
\begin{table}[!htbp]
    \caption{\textbf{Testing Schedule}} \label{Table}
    \begin{tabularx}{\textwidth}{p{3cm}XX}
        \toprule
        \textbf{Date} & \textbf{Task} & \textbf{Team Member}\\
        \midrule
        2017-10-30 & Master Password & Suhavi Sandhu\\
        2017-10-30 & Adding Entries & Shabana Dhayananth\\
        2017-10-30 & Navigation & Joseph Lu\\
        2017-10-04 & Encryption & Suhavi Sandhu\\
        2017-11-04 & Database & Shabana Dhayananth\\
        2017-11-04 & Look and Feel & Joseph Lu\\
        2017-11-07 & Personalization & Suhavi Sandhu\\
        2017-11-07 & Understandibility and Politeness & Shabana Dhayananth\\
        2017-11-07 & Performance & Joseph Lu\\
        2017-11-10 & Operational and Environment Requirement & Suhavi Sandhu\\
        2017-11-10 & Security & Joseph Lu\\
        \bottomrule
    \end{tabularx}
\end{table}

\subsection{Types of Tests}
Refer to Table 5

\begin{table}[!htbp]
    \caption{\textbf{Types of Tests} \label{Table}}
    \begin{tabularx}{\textwidth}{p{3cm}X}
        \toprule
        \textbf{Test Type} & \textbf{Where/when is it applied}\\
        \midrule
        Structural & Will be used on encryption method, database insertion and deletion\\\hline
        Functional & A type of Black box test to test functional requirements\\\hline 
        Unit & For each function in the program. Mostly Functional\\\hline
        System & For testing whole system after integrating database, GUI and encryption\\\hline
        Error Checking & Erroneous password input and missing parameter. Will be conducted near the end\\\hline
        Black box & Testing functionality of product, used primarily for testing existing implementation\\\hline
        White box & Tests internal structures, for testing database\\\hline
        Beta & User testing used for testing non-functional requirements\\
        \bottomrule
    \end{tabularx}
\end{table}

\section{System Test Description}
	
\subsection{Tests for Functional Requirements}

\subsubsection{User Input}
		
\paragraph{Master Password}

\begin{enumerate}

\item{FR-MP-1\\}

Type: Functional, Dynamic, Manual
					
Initial State: Home screen of application, when master password has not been initialized 
					
Input: Password with at least 8 characters, at least one capital and lowercase letter and at least 1 digit.
					
Output: The account is created and the user is brought to the password management page.

How test will be performed: The function that checks that the password satisfies all requirements will be called when the user submits the password. Once the password is submitted and satisfies requirements, the user will be introduced to a new page that is a result of changing user interface frames.
					
					
\item{FR-MP-2\\}

Type: Functional, Dynamic, Manual

Initial State: Home screen of application, when master password has not been initialized

Input: Password that has more than 8 characters, has lower and uppercase but no numbers

Output: The system does not create account and instead prompts with “Password must include at least 1 number”

How test will be performed: The function that checks that the password satisfies all requirements will be called when the user submits the password. If the password does not satisfy requirements, it outputs the appropriate response to what the password is missing.

\item{FR-MP-3\\}

Type: Functional, Dynamic, Manual

Initial State: Home screen of application, when master password has not been initialized 

Input: Password that is not at least 8 characters and does not have at least one capital and lowercase letter and at least 1 digit.

Output: The system does not create account and instead prompts with “Password must be at least 8 characters”

How test will be performed: The function that checks that the password satisfies all requirements will be called when the user submits the password. If the password does not satisfy requirements, it outputs the appropriate response to what the password is missing.

\item{FR-MP-4\\}

Type: Functional, Dynamic, Manual

Initial State: Home screen of application, when master password has not been initialized 

Input: Password has numbers and at least 8 characters but no uppercase letters

Output: The system does not create account and instead prompts with “Password must have upper and lowercase letters”

How test will be performed: The function that checks that the password satisfies all requirements will be called when the user submits the password. If the password does not satisfy requirements, it outputs the appropriate response to what the password is missing.

\item{FR-MP-5\\}

Type: Functional, Dynamic, Manual

Initial State: Home screen of application, when master password has not been initialized 

Input: Empty string

Output: The system does not create account and instead prompts with “Password must not be empty!”

How test will be performed: The function that checks that the password satisfies all requirements will be called when the user submits the password. If the password does not satisfy requirements, it outputs the appropriate response to what the password is missing.

\item{FR-MP-6\\}

Type: Functional, Dynamic, Manual

Initial State: Master password has been initialized and the user is at the Settings page and wants to change the master password

Input: User enters a new master password with at least 8 characters, numbers, upper and lowercase letters and submits it

Output: The system lets the user proceed. The user should be able to access the application with the new updated master password

How test will be performed: The function that checks that the password satisfies all requirements will be called when the user submits the password. If the password satisfies requirements, a function updates the master password in the database.

\item{FR-MP-7\\}

Type: Functional, Dynamic, Manual

Initial State: Master password has been initialized and the user opens application

Input: User enters correct password

Output: The system lets the user proceed. The user should be taken to the password management page where one can add new accounts or view saved account information

How test will be performed: A function will be called to compare the user’s input and the decrypted master password stored in the database. If the password is the same, the window is updated.

\item{FR-MP-8\\}

Type: Functional, Dynamic, Manual

Initial State: Master password has been initialized and the user opens application

Input: User enters incorrect password

Output: The system does not let the user proceed and 

How test will be performed: A function will be called to compare the user’s input and the decrypted master password stored in the database. If the password is the same, the window is updated.

\paragraph{Adding Entries}

\item{FR-AE-1\\}

Type: Functional, Dynamic, Manual

Initial State: User is logged in with master password and wishes to add an entry to the password manager by clicking with ‘Add’ option

Input: For Type, enter what the password is for (i.e Facebook account), for Username enter a valid username and for password enter a valid password

Output: The system adds the entry to the database, upon encrypting it, and the user should be able to see the entry in their list of saved account information.

How test will be performed: Given that a password is provided, the function that encrypts the username and password will be called, thereby encrypting it and the data will be stored in the database. To make it appear, a new label will be added to the window that shows what the user has just added (with the password in asterisks).

\item{FR-AE-2\\}

Type: Functional, Dynamic, Manual

Initial State: User is logged in with master password and wishes to add an entry to the password manager by clicking with ‘Add’ option

Input: For Type, Username and Password, leave fields empty and click ‘Add’

Output: The system does not let user proceed and prompts for a password

How test will be performed: The system uses a function to check that the password is not null and is it is, displays a label prompting user to enter a password

\item{FR-AE-3\\}

Type: Functional, Dynamic, Manual

Initial State: User is logged in with master password and wishes to add an entry to the password manager by clicking with ‘Add’ option

Input: Leave Type and User blank, for password enter a valid password

Output: The system adds the entry to the database, upon encrypting it, and the user should be able to see the entry in their list of saved account information.

How test will be performed: Given that a \textcolor{blue}{name} is provided, the function that encrypts the \sout{username} and password will be called, thereby encrypting it and the data will be stored in the database. To make it appear, a new label will be added to the window that shows what the user has just added \sout{(with the password in asterisks)}.

\subsubsection{Navigation}

\item{FR-N-1\\}

Type: Functional, Dynamic, Manual

Initial State: Password Management screen

Input: User clicks on user manual

Output: User Manual opens and is displayed on the screen

How test will be performed: The application will be opened and the function to make the user manual appear will be called

\item{FR-N-2\\}

Type: Functional, Dynamic, Manual

Initial State: User manual screen

Input: User clicks clicks back button

<<<<<<< HEAD
Output: \sout{Password management screen opens and is displayed on the screen} \textcolor{blue}{Basic instructions and link to user manual are displayed}
=======
Output: Password management screen opens and is displayed on the screen
>>>>>>> d8f26565a1465b0f96ba8356b25db79f079587e0

How test will be performed: The application will be opened and the function to make the password management screen appear will be called

\item{FR-N-3\\}

Type: Functional, Dynamic, Manual

Initial State: User is at password management screen and nothing has been added

Input: User clicks Add button

Output: 3 fields are displayed: Type, Username and Password

How test will be performed: The application will update the interface to show 3 new labels and text fields when the button is clicked

\item{\sout{FR-N-4}\\}

<<<<<<< HEAD
\sout{Type: Functional, Dynamic, \sout{Manual} Automated}
=======
Type: Functional, Dynamic, Manual
>>>>>>> d8f26565a1465b0f96ba8356b25db79f079587e0

Initial State: User is at password management screen and an entry has been added

Input: User clicks Add button

Output: 3 fields are displayed under the existing entry: Type, Username and Password

How test will be performed: The application will update the interface to show 3 new labels and text fields when the button is clicked, keeping the previous entries

\item{FR-N-5\\}

Type: Functional, Dynamic, Manual

Initial State: User has added an entry and wishes to copy the password

Input: User clicks on the ‘Copy’ button beside the password in asterisks

Output: The password is copied and can be pasted into any text inputter

How test will be performed: The application decrypts the password and copies it for the user using the copy() function

\item{FR-N-6\\}

Type: Functional, Dynamic, Manual

Initial State: User is on password management screen and \textcolor{blue}{goes to data entry section}

Input: User clicks on ‘Generate’ button for password

Output: A random password is generated

How test will be performed: The function to generate a random password is called and the password appears in the text field

\item{FR-N-7\\}

Type: Functional, Dynamic, Manual

Initial State: User is on password management screen and \textcolor{blue}{goes to data entry section}

Input: User clicks on ‘Generate’ button twice

Output: A random password is generated and then replaced with the new 

How test will be performed: The function to generate a random password is called and the password appears in the text field. The second time, this text is replaced with the new random password

\subsubsection{Encryption}

\item{FR-E-1\\}

Type: Unit, Dynamic, Automated

Initial State: Encryption function is called

Input: Any string

Output: String gets encrypted

How test will be performed: The encrypted string will be printed to check if it is encrypted

\item{FR-E-2\\}

Type: Unit, Dynamic, Automated

Initial State: Decryption function is called

Input: An already encrypted string

Output: String gets decrypted

How test will be performed: The decrypted string will be printed to see if it matches the original string that was encrypted

\subsubsection{Database}

\item{FR-DB-1\\}

Type: Functional, Dynamic, Manual

Initial State: Empty database

Input: A \textcolor{blue}{query} is created

Output: The \textcolor{blue}{query is created} in the table as its encrypted form with all \textcolor{blue}{unused} fields \textcolor{blue}{as Empty Strings}

How test will be performed: The database table that stores records will be checked to see if the first entry \textcolor{blue}{has ID of 1 and if password does not equals the encrypted value}

\item{FR-DB-2\\}

Type: Functional, Dynamic, Manual

Initial State: \textcolor{blue}{A query} has already been created

Input: User modifies \textcolor{blue}{the first query's username and} password

Output: The updated \textcolor{blue}{query} appears in the table as its encrypted form with \textcolor{blue}{all unused fields as Empty Strings}

How test will be performed: The database table that stores records will be updated and then verified to see if the first entry of the database has updated

\item{FR-DB-3\\}

Type: Functional, Dynamic, Manual

Initial State: \textcolor{blue}{Multiple passwords exists}

Input: \textcolor{blue}{User Deletes the first password}

Output: \textcolor{blue}{The length of the Database decrements by one, and the deleted password is not in the database}

How test will be performed: \textcolor{blue}{The database table that stores records will have one less password. Password with ID of 1 will not exist in the database}

\item{\textcolor{blue}{FR-DB-4}\\}

\textcolor{blue}{Type: Functional, Dynamic, Manual

Initial State: Multiple passwords exists

Input: User gets all passwords, all passwords by type, first password by id, first password by name

Output: No change

How test will be performed: Equal passwords matches with each other.} 
\end{enumerate}

\subsection{Tests for Nonfunctional Requirements}

\subsubsection{Look and Feel Requirements}

\begin{enumerate}

\item{NFR-LFR-1\\}

Type: Structural, Dynamic, Manual

Initial State: Main screen, before logging on

Input: Click enter or button to access another screen (ex. \sout{settings} \textcolor{blue}{after master password entry})

Output: Next screen containing different functions appears

How test will be performed: Test navigation between screens and make sure that functions on each screen are best placed in that grouping using beta testing.

\item{NFR-LFR-2\\}

Type: Structural, Dynamic, Manual

Initial State: Password manager screen, after the user has logged in using their master password

Input: The required fields are account type (for example, email, bank, etc.) and the username and password associated with the account.

Output: If one field is not filled in by the user (empty) a pop-up notification will appear to inform the user that they are missing a required field

How test will be performed: Different combinations of each of the three fields will be omitted intentionally to check if the notification appears in each of those cases.

\item{NFR-LFR-3\\}

Type: Structural, Dynamic, Manual, Unit

Initial State3 Main screen, upon first entry when the user is required to choose a master password for their account

Input: A strong password that contains at least 8 characters consisting of uppercase, lowercase, and numbers

Output: After the enter button is pressed, messages will appear under the input box denoting the missing password criteria.

How test will be performed: A variety of passwords missing a combination of the strong password criteria will be input in order to check if the messages are displayed correctly.

\subsubsection{Personalization Requirements}

\item{NFR-PSR-1\\}

Type: Structural, Dynamic, Manual

Initial State: Password manager screen, after the user has logged in using their master password

Input: The user specifies the account/service type which they wish to add a username and password for

Output: After all \sout{three} \textcolor{blue}{four fields (account type, account name, username and password)} are input, the account type gets stored into the database.

How test will be performed: The account type will be input using normal user data entry so the success of the test case can be verified by ensuring that the input data has been added to the database.

\subsubsection{Understandability and Politeness Requirements}

\item{NFR-UPR-1\\}

Type: Manual

Initial State: Before the application is released

Input: Online user survey with potential symbols that will be used in the product

Output: User feedback on survey

How test will be performed: The user feedback on what the symbols are assumed to represent will allow for the most understandable symbols to be used in the product design.

\subsubsection{Performance}

\item{NFR-PER-1\\}

Type: System, Dynamic, Automated, Unit

Initial State: All required fields of input are filled but the enter button is not yet pressed

Input: User presses enter button

Output: In test report, time elapsed between button press and output display is given.

How test will be performed: An in-code time will be used to ensure that the response time to all user inputs is within PROCESSING-TIME \sout{(2 seconds)}.

\item{NFR-PER-2\\}

Type: System, Dynamic, Automated, Error checking

Initial State: Not all required fields of input are filled but the enter button is not yet pressed

Input: User presses enter button

Output: In test report, time elapsed between button press and error message output display is given.

How test will be performed: An in-code time will be used to ensure that the response time to all user input error is within ERROR-TIME \sout{(2 seconds)}.

\subsubsection{Operational and Environmental Requirements}

\item{NFR-OER-1\\}

Type: System, Dynamic

Initial State: Application is downloaded onto OS but not yet executed.

Input: Execution of system

Output: No unexpected behaviour on OS that is not the OS that the product was developed on.

How test will be performed: All functional test cases will be checked on OSX and Linux OS.

\subsubsection{Security Requirements}

\item{NFR-SR-1\\}

Type: Structural, Dynamic, Automated, Unit

Initial State: Password manager screen, after the user has logged in using their master password

Input: The user specifies the account/service type, a username and a password

Output: Account type, username and encrypted password are stored in database

How test will be performed: Database will be accessed directly to view format of data that was added.
 
\item{NFR-SR-2}

Type: Structural, Dynamic, Automated, Unit

Initial State: Settings screen, after the user is logged in using current master password

Input: User enters new master password

Output: \sout{Password is checked against strength criteria, and if those are met,} new password is saved and old password deleted

How test will be performed: Change master password option will only be displayed if the user is logged in and on the settings page.

\item{NFR-SR-3}

\sout{Type: Structural, Dynamic, Automated, Unit}

\sout{Initial State: User is using application}

\sout{Input: User stops actively using application for a period of time}

\sout{Output: In test report, time elapsed between idle state and automatic logout is given.}

\sout{How test will be performed: An in-code/test timer will be used to measure if the application logs the user out after it is idle for over one minute.}

\end{enumerate}

\subsection{Traceability Table Between Test Cases and Requirements}
Refer to Table 6 and Table 7
\begin{table}[!htbp]
    \caption{\textbf{Traceability table for Functional Requirements}} \label{Table}
    \begin{tabularx}{\textwidth}{p{2cm}Xp{2cm}X}
        \toprule
        \textbf{Req ID} & \textbf{Description} & \textbf{Priority} & \textbf{Test Cases}\\
        \midrule
        FR1  & The executable Python code will create a user interface window & High & FR-MP-1\\\hline
        FR2  & Upon execution, the program will have a connection to a local database & High & FR-DB-1, FR-DB-2, FR-DB-3, FR-DB-4\\\hline
        FR3  & The user must be able create a master password & High & FR-MP-1, FR-MP-2, FR-MP-3, FR-MP-4-, FR-MP-5\\\hline
        FR4  & The user must be able to enter a master password & High & FR-MP-7, FR-MP-8\\\hline
<<<<<<< HEAD
        FR5  & The user must be able to add a new entry to the manager & High & FR-AE-1, FR-AE2, FR-AE-3, \sout{FR-N-3}, \sout{FR-N-4}\\\hline
        FR6  & The user should be able to generate a random password  & High & FR-N-6, FR-N-7, \textcolor{blue}{FR-PWG-1}\\\hline
=======
        FR5  & The user must be able to add a new entry to the manager & High & FR-AE-1, FR-AE2, FR-AE-3, FR-N-3, FR-N-4\head\\hline
        FR6  & The user should be able to generate a random password  & High & FR-N-6, FR-N-7\\\hline
>>>>>>> d8f26565a1465b0f96ba8356b25db79f079587e0
        FR7  & The user interface must have a link to the user manual & High & FR-N-1, FR-N-2\\\hline
        FR8  & When user wants to copy or view a password, the software should decrypt it & High & FR-N-5, FR-E-2\\\hline
        FR9  & The application should have buttons to directly copy username and password & High & FR-N-5\\\hline
        FR10 & The user should be able to change their master password  & High & FR-MP-6\\\hline
        FR11 & When a user adds a new entry, the software should encrypt the input & High & FR-E-1\\
    
        \bottomrule
    \end{tabularx}
\end{table}

\begin{table}[!htbp]
    \caption{\textbf{Traceability table for Non-functional Requirements}} \label{Table}
    \small
    \begin{tabularx}{\textwidth}{p{2cm}Xp{2cm}X}
        \toprule
        \textbf{Req ID} & \textbf{Description} & \textbf{Priority} & \textbf{Test Cases}\\
        \midrule
        
        NFR1  & Different screens for different functions & High & NFR-LFR-1, beta testing\\\hline
        NFR2  & Pop-up notification when requirements are missing & High & NFR-LFR-2\\\hline
        NFR3  & Clearly display master password criteria before processing & High & NFR-LFR-3\\\hline      
        NFR4  & The product will have a minimalistic aesthetic that is easy for users to find their way around the application & Med & Beta testing\\\hline
        NFR5  & Create a modern and functional feel for the product. Rationale: The user must want to use this app & Low & Beta testing\\\hline
        NFR6  & The product shall be intuitive and easy to understand and use. & High & Beta testing\\\hline
        NFR7  & The product shall allow for the user to be able to categorize their passwords based on the service & Low & NFR-PSR-1\\\hline
        NFR8  & The user shall be able to navigate their computer. & Low & Beta testing\\
        
        \midrule 
    \end{tabularx}
\end{table}

\begin{table}[!htbp]
    \begin{tabularx}{\textwidth}{p{2cm}Xp{2cm}X}
        \midrule

        NFR9  & The product shall use symbols and words that are naturally understandable & Low & NFR-UPR-1\\\hline
        NFR10 & The product shall hide the details of its construction to the user. & High & Beta testing\\\hline
        NFR11 & The product shall respond to user input within PROCESSING-TIME & High & NFR-PER-1\\\hline
        NFR12 & The application shall be responsive and when not all requirements for a process are met, error message within ERROR-TIME & High & NFR-PER-2\\\hline
        NFR13 & The program shall run on Windows, OSX and Linux OS & Low & NFR-OER-1\\\hline
        NFR14 & The product will be available as open source. & Low & N/A\\\hline
        NFR15 & The product shall only store account types, usernames and encrypted passwords. & High & NFR-SR-1\\\hline
        NFR16 & Master password cannot be modified unless user is already logged in & Hight & \sout{NFR-SR-2}\\\hline
        NFR17 & The product shall contain security measures in order to prevent the data from being accessed directly from the user’s machine & High & NFR-SR-3\\\hline
        NFR18 & The product shall not distribute the user’s personal information to third parties & High & N/A\\

        \bottomrule

    \end{tabularx}
\end{table}

\section{Tests for Proof of Concept}

Testing for Proof of Concept will be unit and structural test what will be used for a Proof of Concept demo. These tests are to test the functionality of the program. The areas that the proof of concept testing will focus on are User Interface, Encryption and Database.

\subsection{Master Password}

\begin{enumerate}

\item{POC-1\\}

Type: Functional, Dynamic, Manual

Initial State: GUI is run for the first time

Input: A valid password is entered under Master Password

Output: User is brought to the password management screen

How test will be performed: For the proof of concept, the master password will not be saved in the database or encrypted as those are independent during the POC. Instead, a function that checks password logic will allow the screen to update to the next screen if password criteria is satisfied

\item{POC-2\\}
					
Type: Functional, Dynamic, Manual

Initial State: GUI is run for the second time after master password is initialized

Input: The correct password is entered under Master Password

Output: User is brought to password management screen

How test will be performed: For the proof of concept, the master password from before will be saved and upon execution for the second time and forwards, it will proceed only if the user types in the correct password

\item{POC-3\\}

Type: Functional, Dynamic, Manual

Initial State: Password management screen

Input: Valid account information is entered for type, username and password

Output: Entry appears as label on screen 

How test will be performed: For the proof of concept, the input is checked for criteria satisfaction (is the password not null), if these requirements are satisfied, the entry is displayed on the GUI frame

\item{POC-4\\}

Type: Unit, Dynamic, Automated

Initial State: Encryption function is called

Input: Any string

Output: String gets encrypted

How test will be performed: The encrypted string will be printed to check if it is encrypted

\item{POC-5\\}

Type: Unit, Dynamic, Automated

Initial State: Decryption function is called

Input: An already encrypted string

Output: String gets decrypted

How test will be performed: The decrypted string will be printed to see if it matches the original string that was encrypted

\item{POC-6\\}

Type: Functional, Dynamic, Manual

Initial State: Connection to database has not been established

Input: Connect to database (Database.Connect())

Output: Can print contents through command line

How test will be performed: For the proof of concept, the database is independent of the GUI and encryption method. For this reason, after the connection is established, it can be verified by printing the table contents through the command line

\item{POC-7\\}

Type: Functional, Dynamic, Manual

Initial State: Connection to database has been established

Input: Insert into database

Output: New content shows up when printing through command line

How test will be performed: For the proof of concept, the database is independent of the GUI and encryption method. For this reason, on insertion of data, it can be checked that the data was indeed inserted by printing the table contents through the command line

\end{enumerate}

\section{Comparison to Existing Implementation}	
				
The existing implementation, Padlock, which our implementation is based on, is very similar in nature in terms of user interface. Therefore, the main areas of testing that can be compared to the existing product are user input and navigation. These tests will be valuable in improving the product and testing parallely to see what our implementation lacks or could do differently. Relevant test cases that can be used for comparison are:\\

Functional Tests
\begin{description} 
    \item User Input
    \begin{description}
        \item[$\bullet$] FR-MP-1 - Creation of Master Password
        \item[$\bullet$] FR-MP-5 - Empty string for master password creation
        \item[$\bullet$] \sout{FR-MP-6 - Changing master password from settings}
        \item[$\bullet$] FR-MP-7 - Logging in using master password
        \item[$\bullet$] FR-MP-8 - Attempting to log in using incorrect master password
        \item[$\bullet$] FR-AE-1 - Add an entry for account with type, username and password
    \end{description}
    \item Navigation
    \begin{description}
        \item[$\bullet$] FR-N-5 - Copy password from existing entry
        \item[$\bullet$] FR-N-6 - Generate random password for account entry
    \end{description}
\end{description}

Nonfunctional Tests
\begin{description}
    \item Operational and Environmental Requirements
    \begin{description}
          \item[$\bullet$] NFR-OER-1 - Program compatible to Windows, OSX and Linux OS
    \end{description}
    \item Personalization Requirements
    \begin{description}
          \item[$\bullet$] NFR-PSR-1 - User can categorize usernames and passwords based on account type
    \end{description}
    \item{Performance Requirements}
    \begin{description}
          \item[$\bullet$] \sout{NFR-PER-1 - Output response to user input shall happen in a 2 second timeframe}
          \item[$\bullet$] \sout{NFR-PER-2 - Error response to user input shall happen in a 2 second timeframe}
    \end{description}
    \item{Security}
    \begin{description}
          \item[$\bullet$] NFR-SR-3 - Logs user out after one minute of inactivity for security
    \end{description}
\end{description}

\section{Unit Testing Plan}
		
\subsection{Unit testing of internal functions}
		
Unit testing will be executed using PyUnit and all functions of the application will be tested. In functional requirements there are few unit tests because the tests are heavily focused on user interface. However, to test that internal functions work as expected, the following tests will be executed.

\begin{itemize}
    \item \texttt{checkPassword(password)} - checks if the password satisfies criteria
    \item \texttt{insert(id, type, username, password)} - inserts id of entry, type, username and encrypted password
    \item \texttt{updateMasterPassword()} - updates master password in database if it is modified    
    \item \texttt{username(id)} - returns username from database
    
    \item \texttt{password(id)} - returns password from database, which is later decrypted
    \item \texttt{copy(id)} - gets password at specific id, decrypts it and returns
    \item \texttt{generateRand()} - generates a random password
    \item \texttt{generKey()} - generates key for encryption
    \item \texttt{cryptEncode(key f, String password)} - encrypts input password using Fernet key
    \item \texttt{decrypt(String encryptedPassword)} - decrypts encrypted password using key
\end{itemize}

\subsection{Unit testing of output files}		

Below is a description of the output files that will be needed for unit testing and how the functions that are being unit tested will use the output files. Not all functions require an output file for unit testing. Coverage metrics are also discussed.\\

\texttt{passwords.txt}
\begin{itemize}
    \item \texttt{passwords.txt} will have a list of passwords that may or may not satisfy password criteria 
    \item At the end of the file there will be a Boolean array that tells you if the passwords satisfy the criteria or not
    \item Used for checkPassword() to test if passwords are being evaluated correctly. Coverage metric will be comparison of unit test output to the Boolean array at the end of file
    \item Used for updateMasterPassword() to change master password consecutively.
    \item Used for generKey(), cryptEncode(key f, String password), decrypt(String encryptedPassword) to encrypt and decrypt passwords
\end{itemize}

\texttt{tuples.csv}
\begin{itemize}
    \item Will have records to be inserted into the database
    \item used for insert(id, type, username, password) to insert into database
    \item used for username(id) and password(id) to have something in database to return
    \item Used for copy(id) to see if each password entry gets copied
\end{itemize}

\texttt{copies.txt}
\begin{itemize}
    \item The output file which will be compared to all password entries in tuples.csv
    \item Used for copy(id) to check if the passwords in tuples.csv were copied correctly to copies.txt
\end{itemize}

\texttt{generated1.txt} and \texttt{generated2.txt}
\begin{itemize}
    \item Used for generateRand() in which both files will be filled with generated passwords and the coverage metric will be to verify how similar the two are. This will be loosely done by comparing each line in generated1.txt to all lines in generated2.txt. Fewest similarities are ideal 
\end{itemize}

\bibliographystyle{plainnat}

\bibliography{SRS}

\newpage

\end{document}
