\documentclass[12pt, titlepage]{article}

\usepackage{booktabs}
\usepackage{tabularx}
\usepackage{hyperref}
\hypersetup{
    colorlinks,
    citecolor=black,
    filecolor=black,
    linkcolor=red,
    urlcolor=blue
}
\usepackage[round]{natbib}

\title{SE 3XA3: Test Report\\Title of Project}

\author{Team \#, Team Name
		\\ Suhavi Sandhu sandhs11
		\\ Shabana Dhayananth dhayanas
		\\ Joseph Lu luy89
}

\date{\today}

\begin{document}

\maketitle

\pagenumbering{roman}
\tableofcontents
\listoftables
\listoffigures

\begin{table}[bp]
\caption{\bf Revision History}
\begin{tabularx}{\textwidth}{p{3cm}p{2cm}X}
\toprule {\bf Date} & {\bf Version} & {\bf Notes}\\
\midrule
2017/12/06 & 1.0 & Creation\\
\bottomrule
\end{tabularx}
\end{table}

\newpage

\pagenumbering{arabic}

This document discusses the results from testing the product, PasswordProtectionProgram, 
any changes that were implemented as a result of testing and the extensiveness of the 
tests performed.

\section{Functional Requirements Evaluation}

\subsection{User Input}

\paragraph{Master Password}
These test cases verify that the functions related to the master password, a critical part of 
the system, behave as intended.  
\newline
\\
	The following test cases assume that a master password has not yet been initialized and test for the creation of the master password.

	FR-MP-1: A valid input of at least characters having upper and lowercase as well as numbers should let user proceed.
	Result: Test pass, user is allowed to proceed.
	
	FR-MP-2 to FR-MP-5: Invalid input for master password should display appropriate error message.
	Results: All tests pass and display appropriate error message.
	
	The following test cases assume that a master password has already been initialized and test for logging in with the master password.
	
	FR-MP-6: Upon entry of an empty string, the system should display an error message.
	Result: This test originally passed, however, due to a bug found during the testing for ‘update entry’, this test was omitted. 
	
	FR-MP-7, FR-MP-8: Depending on whether input is correct or incorrect, system lets user proceed or displays error message.
	Result: Tests pass
	
\paragraph{Adding Entries}
These tests verify that users are able to add new entries to the database, assuming a connection to the database is already established and user is logged into application.

	FR-AE-1, FR-AE-2, FR-AE-3: Originally, the goal of these test cases was to ensure that when adding an entry,
	a user cannot add an empty password. Upon testing, it became clear that an empty password should be allowed,
	in the case that the user has not come up with one. Therefore, the test was changed to verify that a Name,
	rather than a password, should be required.
	Results: All test cases pass, allowing users to add entries as long as name is provided, else error message displayed.
	
\paragraph{Navigation}
The underlying goal for the product was to make it more userfriendly than the original. In doing so,
great care was given to the tests designated under navigation. The following tests verify the functionality
of the user manual.

FR-N-1: Link to user manual works.
Results: A pdf viewer opens up with the user guide.

FR-N-2: 

	
	
\section{Nonfunctional Requirements Evaluation}

\subsection{Usability}
		
\subsection{Performance}

\subsection{etc.}
	
\section{Comparison to Existing Implementation}	

This section will not be appropriate for every project.

\section{Unit Testing}

\section{Changes Due to Testing}

\section{Automated Testing}
		
\section{Trace to Requirements}
		
\section{Trace to Modules}		

\section{Code Coverage Metrics}

\bibliographystyle{plainnat}

\bibliography{SRS}

\end{document}